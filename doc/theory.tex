\documentclass[amsmath,%draft,
amssymb,prb,superscriptaddress]{revtex4}
\usepackage{graphicx}
\usepackage{lpic}
\usepackage{amsmath}
\usepackage{caption}
\usepackage{hyperref}

\begin{document}

\title{On the Hubbard Model in the Mean Field approximation}

\author{S. Sanz}
\affiliation{\mbox{Donostia International Physics Center (DIPC) -- UPV/EHU, 20018 San
Sebasti\'an, Spain}}

\author{T. Frederiksen}
\affiliation{\mbox{Donostia International Physics Center (DIPC) -- UPV/EHU, 20018 San
Sebasti\'an, Spain}}
\affiliation{IKERBASQUE, Basque Foundation for Science, E-48011, Bilbao, Spain}

\date{\today}



\maketitle 


\section{Inclusion of electron correlations: Hubbard model}

\begin{equation}\label{eq:Hubbard_Hamiltonian}
H = -\sum_{ij\sigma}t_{ij\sigma}c^{\dagger}_{i\sigma}c_{j\sigma} + U\sum_{i}n_{i\uparrow}n_{i\downarrow}
\end{equation}

\subsection{Exact solution: two sites}
Despite the aparent simplicity of the model it is very hard to solve for most of the cases. In fact, there are only two particular cases in which analytical solutions are found: the one dimensional [ref] chain and the Hubbard dimer.
 


\subsection{Mean Field Approximation}

In this approximation it is assumed that the density operator is very close to its expected value, that is $n_{i\sigma} \approx \langle n_{i\sigma}\rangle$. This means that the densities fluctuations (defined as $\delta_{i\sigma} =n_{i\sigma} - \left\langle n_{i\sigma}\right\rangle$) are considered to be small quantities compared to $\left\langle n_{i\sigma}\right\rangle$. The two particle operator in this case would approximated by:

\begin{equation}\label{eq:MFH}
\begin{split}
n_{i\uparrow}n_{i\downarrow} & = \left( \delta_{i\uparrow}+\left\langle n_{i\uparrow}\right\rangle\right) 
\left( \delta_{i\downarrow}+\left\langle n_{i\downarrow}\right\rangle\right) = \\
 & = \delta_{i\uparrow}\delta_{i\downarrow} +  \left\langle n_{i\uparrow}\right\rangle\delta_{i\downarrow} + \delta_{i\uparrow}\left\langle n_{i\downarrow}\right\rangle \approx \\
 & \approx \left\langle n_{i\uparrow}\right\rangle n_{i\downarrow} + \left\langle n_{i\downarrow}\right\rangle n_{i\uparrow} - 2\left\langle n_{i\uparrow}\right\rangle\left\langle n_{i\downarrow}\right\rangle
\end{split}
\end{equation}

Where it has been approximatied that $\delta_{i\uparrow}\delta_{i\downarrow}\approx 0$ compared to the rest of elements in Eq. (\ref{eq:MFH}). 

The code


\subsection{Linear Combination of Atomic Orbitals}

%Calculation of spin density when non-orthogonal basis is set (taken from Celia Fonseca Guerra et al, J. comput Chem 25: 189-210, 2004):
%
%\begin{equation}\label{eq:Charge_overlap_matrix}
%n_{\mu} = D_{\mu\mu}S_{\mu\mu} + \frac{1}{2}\sum_{\nu(\neq\mu)}\left(D_{\mu\nu}S_{\mu\nu} + D_{\nu\mu}S_{\nu\mu} \right)
%\end{equation}
% This is the same as computing:
%
%\begin{equation}
%\left\langle n_{i\sigma}\right\rangle = \sum_{j}S_{ij\sigma}D_{ij\sigma}
%\end{equation}

In the more general case the problem to solve is the generalized eigenvalue problem:

\begin{equation}
\textbf{H}\Psi_{\alpha} = E_{\alpha}\textbf{S}\Psi_{\alpha}
\end{equation}

Notation: latin indices will represent site indices (atomic positions) while greek indices stand for energy levels.

The wavefunction  in real space:

\begin{equation}\label{eq:TBWF}
\Psi_{\alpha}(\textbf{r}) = \sum_{i} c_{\alpha i}\phi(\textbf{R}_{i})
\end{equation}

The overlap matrix is defined as:

\begin{equation}
S_{ij} = \left\langle\phi(\textbf{R}_{i}) \right|\left. \phi(\textbf{R}_{j}) \right\rangle
\end{equation}

The system satisfies:

\begin{equation}
S_{ij}c_{j\alpha}c^{*}_{\alpha l} = \delta_{il} \Rightarrow \textbf{C}\textbf{C}^{\dagger} = \textbf{S}^{-1}
\end{equation}

Where $\textbf{C}$ is the matrix of eigenvectors whose components are $c_{\alpha i}$.\\

\medskip
We define the Density operator $\hat{\rho}$ as:

\begin{equation}
\hat{\rho_{\sigma}} = \sum_{\alpha}f_{\alpha\sigma} \left| \Psi_{\alpha\sigma}\right\rangle \left\langle \Psi_{\alpha\sigma} \right|
\end{equation}

Where $f_{\alpha\sigma}$ is the weight of state $\alpha$ for spin $\sigma$ (Fermi-Dirac distribution).\\

Since we describe the Hamiltonian (and eigenstates) in the basis of atomic sites, it seems natural to define the Density operator in terms of this basis:
 
\begin{equation}\label{eq:DMatrix}
D_{ij\sigma} = \sum_{\alpha}^{occ}f_{\alpha\sigma}c_{i \alpha\sigma}c^{*}_{j \alpha\sigma}
\end{equation}


The occupations will be found as the charge associated with oribtal $i$ explicitly, that is, the ``Mulliken'' populations $n_{i}$ (ref of Hanckock paper?):

\begin{equation}
n_{i\sigma} = \sum_{\alpha}f_{\alpha\sigma}\sum_{j}c^{\alpha}_{i\sigma}c^{*\alpha}_{j\sigma}S_{ij} = \sum_{j}D_{ij\sigma}S_{ij} =  (\textbf{D}_{\sigma}\textbf{S})_{ii}
\end{equation}

$\sigma$ index in overlap matrix $\textbf{S}$ is not present because the overlap matrix is the same for both spin components.
The density matrix $D$ is defined in (\ref{eq:DMatrix}), with $c_{i\mu}$ the coefficients of the wavefunction (see eq. (\ref{eq:TBWF})).\\

\medskip

Tipically the tight-binding model we use is described in an orthogonal basis set. This means that $\langle \phi (\mathbf{R}_{i}) | \phi (\mathbf{R}_{j}) \rangle = \delta_{ij}$. Replacing the overlap matrix by $S_{ij}=\delta_{ij}$  we recover:

\begin{equation}
n_{i\sigma} = \sum_{\alpha}f_{\alpha\sigma}\left| c^{\alpha}_{i\sigma}\right|^{2}
\end{equation}


No summation convention was assumed in any of the written equations.\\

This can be extended to the $k$-space when the system is periodic [ref to Segall paper]:


\begin{equation}
\textbf{H}(\textbf{k})\Psi_{\alpha}(\textbf{k}) = E_{\alpha}(\textbf{k})\textbf{S}(\textbf{k})\Psi_{\alpha}(\textbf{k})
\end{equation}

\begin{equation}
n_{i} = \sum_{\textbf{k}}\sum_{\alpha}\sum_{j}c^{\alpha}_{i}(\textbf{k})c^{*\alpha}_{j}(\textbf{k})S_{ij}(\textbf{k})
\end{equation}


It can be seen that $N_{\sigma}=\sum_{i}n_{i\sigma}$.


\subsection{Lieb's theorem}

In bipartite lattices the Lieb's theorem [ref] applies: the total spin of the ground-state is proportional to the imbalance between A and B lattices.

\begin{equation}\label{eq:Lieb_theorem}
S = \frac{1}{2}\left|N_{A}-N_{B}\right|
\end{equation}


\section{Example 1: Finite molecules}

\subsection{Singlet GS}



\subsection{Triplet GS}
Triangulane


\section{Example 2: Periodic structures}

\subsection{Bandstructure: Comparison to DFT}


\section{References}


\end{document}