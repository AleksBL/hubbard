\documentclass[amsmath,%draft,
amssymb,prb,superscriptaddress]{revtex4}
\usepackage{graphicx}
\usepackage{lpic}
\usepackage{amsmath}
\usepackage{caption}
\usepackage{hyperref}

\begin{document}

\title{On the Hubbard Model in the Mean Field approximation}

\author{S. Sanz}
\affiliation{\mbox{Donostia International Physics Center (DIPC) -- UPV/EHU, 20018 San
Sebasti\'an, Spain}}

\author{T. Frederiksen}
\affiliation{\mbox{Donostia International Physics Center (DIPC) -- UPV/EHU, 20018 San
Sebasti\'an, Spain}}
\affiliation{IKERBASQUE, Basque Foundation for Science, E-48011, Bilbao, Spain}

\date{\today}



\maketitle 


\section{Inclusion of electron correlations: Hubbard model}

\begin{equation}\label{eq:Hubbard_Hamiltonian}
H = -\sum_{ij\sigma}t_{ij\sigma}c^{\dagger}_{i\sigma}c_{j\sigma} + U\sum_{i}n_{i\uparrow}n_{i\downarrow}
\end{equation}

\subsection{Exact solution: two sites}
Despite the aparent simplicity of the model it is very hard to solve for most of the cases. In fact, there are only two particular cases in which analytical solutions are found: the one dimensional [ref] chain and the Hubbard dimer.
 


\subsection{Mean Field Approximation}

In this approximation it is assumed that the densities fluctuation are negligible, so the two particle operator is approximated by:

\begin{equation}
n_{i\uparrow}n_{i\downarrow} = \left(n_{i\uparrow} - \left\langle n_{i\uparrow}\right\rangle \right)\left( n_{i\downarrow} - \left\langle n_{i\downarrow}\right\rangle  \right) =
\end{equation}

The code


\subsubsection{Calculations}

%Calculation of spin density when non-orthogonal basis is set (taken from Celia Fonseca Guerra et al, J. comput Chem 25: 189-210, 2004):
%
%\begin{equation}\label{eq:Charge_overlap_matrix}
%n_{\mu} = D_{\mu\mu}S_{\mu\mu} + \frac{1}{2}\sum_{\nu(\neq\mu)}\left(D_{\mu\nu}S_{\mu\nu} + D_{\nu\mu}S_{\nu\mu} \right)
%\end{equation}
% This is the same as computing:
%
%\begin{equation}
%\left\langle n_{i\sigma}\right\rangle = \sum_{j}S_{ij\sigma}D_{ij\sigma}
%\end{equation}

In the more general case the problem to solve is:

\begin{equation}
\textbf{H}\Psi_{\alpha} = E_{\alpha}\textbf{S}\Psi_{\alpha}
\end{equation}

Notation: latin indices will represent site indices (atomic positions) while greek indices stand for energy levels.

The wavefunction  in real space:

\begin{equation}\label{eq:TBWF}
\Psi_{\alpha}(\textbf{r}) = \sum_{i} c_{\alpha i}\phi(\textbf{R}_{i})
\end{equation}

The overlap matrix is defined as:

\begin{equation}
S_{ij} = \left\langle\phi(\textbf{R}_{i}) \right|\left. \phi(\textbf{R}_{j}) \right\rangle
\end{equation}

The system satisfies:

\begin{equation}
S_{ij}C_{jk}C^{\dagger}_{kl} = \delta_{il} \Rightarrow \textbf{C}\textbf{C}^{\dagger} = \textbf{S}^{-1}
\end{equation}

Where $C_{i\alpha} = c_{i\alpha}$ is the matrix of eigenvectors.

%

The occupations will be find as the charge associated with oribtal $i$ explicitly, that is, the ``Mulliken'' populations $n_{i}$ (ref of Hanckock paper?):

\begin{equation}
n_{i\sigma} = \sum_{\alpha}f_{\alpha\sigma}\sum_{j}c^{\alpha}_{i\sigma}c^{*\alpha}_{j\sigma}S_{ij} = \sum_{j}D_{ij\sigma}S_{ij} =  (\textbf{D}_{\sigma}\textbf{S})_{ii}
\end{equation}

$\sigma$ index in overlap matrix $\textbf{S}$ is not present because the overlap matrix is the same for both spin components.
The density matrix $D$ is defined in (\ref{eq:DMatrix}), with $c_{i\mu}$ the coefficients of the wavefunction (see eq. (\ref{eq:TBWF})):

\begin{equation}\label{eq:DMatrix}
D_{ij\sigma} = \sum_{\alpha}^{occ}f_{\alpha\sigma}c_{i \alpha\sigma}c^{*}_{j \alpha\sigma}
\end{equation}


In the more usual orthogoanl tight binding, replacing by $S_{ij}=\delta_{ij}$  we recover:

\begin{equation}
n_{i\sigma} = \sum_{\alpha}f_{\alpha\sigma}\left| c^{\alpha}_{i\sigma}\right|^{2}
\end{equation}


No summation convention was assumed in any of the written equations.\\

This can be extended to the $k$-space when the system is periodic [ref to Segall paper]:


\begin{equation}
\textbf{H}(\textbf{k})\Psi_{\alpha}(\textbf{k}) = E_{\alpha}(\textbf{k})\textbf{S}(\textbf{k})\Psi_{\alpha}(\textbf{k})
\end{equation}

\begin{equation}
n_{i} = \sum_{\textbf{k}}\sum_{\alpha}\sum_{j}c^{\alpha}_{i}(\textbf{k})c^{*\alpha}_{j}(\textbf{k})S_{ij}(\textbf{k})
\end{equation}


It can be seen that $N_{\sigma}=\sum_{i}n_{i\sigma}$.


\subsection{Lieb's theorem}

In bipartite lattices the Lieb's theorem [ref] applies: the total spin of the ground-state is proportional to the imbalance between A and B lattices.

\begin{equation}\label{eq:Lieb_theorem}
S = \frac{1}{2}\left|N_{A}-N_{B}\right|
\end{equation}


\section{Example 1: Finite molecules}

\subsection{Singlet GS}



\subsection{Triplet GS}
Triangulane


\section{Example 2: Periodic structures}

\subsection{Bandstructure: Comparison to DFT}


\section{References}


\end{document}