\documentclass[twocolumn,amsmath,%draft,
amssymb,prb,superscriptaddress]{revtex4}
\usepackage{graphicx}
\usepackage{standalone}
\usepackage{lpic}
\usepackage{pgf}
\usepackage{tikz}

\newcommand{\Eqref}[1]{Eq.~(\ref{#1})}
\newcommand{\Figref}[1]{Fig.~\ref{#1}}
\newcommand{\Secref}[1]{Sec.~\ref{#1}}


\begin{document}

\title{Triangulene}

\author{S. Sanz}
\affiliation{\mbox{Donostia International Physics Center (DIPC) -- UPV/EHU, 20018 San
Sebasti\'an, Spain}}

\author{T. Frederiksen}
\affiliation{\mbox{Donostia International Physics Center (DIPC) -- UPV/EHU, 20018 San
Sebasti\'an, Spain}}
\affiliation{IKERBASQUE, Basque Foundation for Science, E-48011, Bilbao, Spain}

\date{\today}

\begin{abstract}
We analyze the extended triangulene geometry with MFH using up to third nearest neighbors model ($t_{1}=-2.7$ eV, $t_{2}=-0.2$ eV, $t_{3}=-0.18$ eV). We show the spin polarization of the ground state ($S_{z}=1$) for $ U=3.0$ eV, the wavefunctions for $U=0$ and $3.0$ eV, and the dependency of the different magnetic states (different $S_{z}$) as a function of the Coulomb repulsion parameter $U$. We also perform a symilar analysis for the extended-triangulene dimer with a pentagon. The ground state of the dimer also appears for $S_{z}=1$. We also show the effect of H-passivation on different edge C-atomic sites for both triangulene and dimer molecules.
\end{abstract}

\maketitle

\begin{figure*}
	\scalebox{.6}{\documentclass{standalone}

\usepackage{graphicx}
\usepackage{hyperref}
\usepackage{epsfig}
\usepackage{color}
\usepackage{psfrag}
\usepackage{lpic}
\usepackage{tikz}
\usetikzlibrary{decorations.pathmorphing}

\tikzset{
dash/.style={line width=.5pt,dash pattern=on 2pt off 1.5pt}
         }% end of tikzset

%%%%%%%%%%%%%%%%%%%%%%%%%%%%%%%%%%%%%%%%%%%%%%%%%%%%%%%%%%%%%%%%%%%%%%%%%%%%%%

\begin{document}
\centering


\begin{tikzpicture}[>=stealth]

\node [inner sep=0pt] at (-0.5, 0) {\includegraphics[height=4cm]{../3-blocks/AFM-pol-300}};
\node [inner sep=0pt] at (5.5, 0) {\includegraphics[height=4cm]{../3-blocks/FM-pol-300}};
\node [inner sep=0pt] at (2.2, -5.5) {\includegraphics[height=7cm]{../3-blocks/FM-AFM}};

%\node [inner sep=0pt] at (10, 0) {\includegraphics[height=4cm]{../3-blocks-pentagon/AFM-pol-300}};
%\node [inner sep=0pt] at (15, 0) {\includegraphics[height=4cm]{../3-blocks-pentagon/FM-pol-300}};
%\node [inner sep=0pt] at (12.2, -5.5) {\includegraphics[height=7cm]{../3-blocks-pentagon/FM-AFM}};
%
%

\node at (-2.6, 2){\large \textbf{a}};
\node at (3.2, 2){\large \textbf{b}};
\node at (-2.2, -2.){\Large \textbf{c}};

%\node at (-2.75, -4.75){\Large \textbf{d}};
%\node at (5.75, -4.75){\Large \textbf{e}};
%\node at (14.25, -4.75){\Large \textbf{f}};
%

\end{tikzpicture}


\end{document}
%%%%%%%%%%%%%%%%%%%%%%%%%%%%%%%%%%%%%%%%%%%%%%%%%%%%%%%%%%%%%%%%%%%%%%%%%%%%%%%
}
	\caption{\textbf{a-c} Spin polarization of the ground state of the extended triangulene with \textbf{a} two and \textbf{b} one H-atoms and \textbf{c} without H-passivated edge C-atomic sites obtained with $U=3.0$ eV. \textbf{d-f} Density of states at the HOMO energy of the ground state of each molecule from the top row obtained with $U=3.0$ eV.}
\end{figure*}

\begin{figure*}
	\scalebox{.7}{\documentclass{standalone}

\usepackage{graphicx}
\usepackage{hyperref}
\usepackage{epsfig}
\usepackage{color}
\usepackage{psfrag}
\usepackage{lpic}
\usepackage{tikz}
\usetikzlibrary{decorations.pathmorphing}

\tikzset{
dash/.style={line width=.5pt,dash pattern=on 2pt off 1.5pt}
         }% end of tikzset

%%%%%%%%%%%%%%%%%%%%%%%%%%%%%%%%%%%%%%%%%%%%%%%%%%%%%%%%%%%%%%%%%%%%%%%%%%%%%%

\begin{document}
\centering


\begin{tikzpicture}[>=stealth]
\node [inner sep=0pt] at (0., 0) {\includegraphics[height=10cm]{../U0_spectrum}};
\node [inner sep=0pt] at (0, -10.0) {\includegraphics[height=10cm]{../U300_spectrum}};

\node at (13.0, 5.7){\Huge $U=0$};

\node [inner sep=0pt] at (10.3, 2) {\includegraphics[height=6cm]{../U0_state17_up}};
\node [inner sep=0pt] at (16.5, 2) {\includegraphics[height=6cm]{../U0_state18_up}};

\node at (13, -2.2){\Huge $U=3.0$ eV};

\node [inner sep=0pt] at (10.3, -6) {\includegraphics[height=6cm]{../U300_state18_up}};
\node [inner sep=0pt] at (16.5, -6) {\includegraphics[height=6cm]{../U300_state19_up}};
\node [inner sep=0pt] at (10.3, -12) {\includegraphics[height=6cm]{../U300_state16_dn}};
\node [inner sep=0pt] at (16.5, -12) {\includegraphics[height=6cm]{../U300_state17_dn}};

%%%%%%%%%%%%%%%%%%%%%%%%%%%%%%%%%%%%% Draw symmetry axis %%%%%%%%%%%%%%%%%%%%%%%%%%%

\node at (-6.25, 5.35){\huge \textbf{a}};
\node at (-6.25, -5.35){\huge \textbf{b}};

\node at (8.25, 4.35){\huge \textbf{c}};
\node at (14.45, 4.35){\huge \textbf{d}};

\node at (8.25, -3.75){\huge \textbf{d}};
\node at (14.45, -3.75){\huge \textbf{e}};


\node at (8.25, -9.85){\huge \textbf{g}};
\node at (14.45, -9.85){\huge \textbf{h}};

\end{tikzpicture}


\end{document}
%%%%%%%%%%%%%%%%%%%%%%%%%%%%%%%%%%%%%%%%%%%%%%%%%%%%%%%%%%%%%%%%%%%%%%%%%%%%%%%
}
	\caption{Spatial localization of each state of the extended-triangulene molecule computed as $\eta_{\alpha\sigma}=\int dr|\psi_{\alpha\sigma}|^{4}$ obtained with \textbf{a} $U=0$ and \textbf{b} $U=3.0$ eV. \textbf{c, d} HOMO and LUMO wavefunction spatial distributions for $U=0$. \textbf{e-h} HOMO and LUMO spatial distributions of the ground state ($S_{z}=1$) for \emph{up}- and \emph{down}-electrons obtained with $U=3.0$ eV.}
\end{figure*}

\begin{figure*}
	\scalebox{.5}{\documentclass{standalone}

\usepackage{graphicx}
\usepackage{hyperref}
\usepackage{epsfig}
\usepackage{color}
\usepackage{psfrag}
\usepackage{lpic}
\usepackage{tikz}
\usetikzlibrary{decorations.pathmorphing}

\tikzset{
dash/.style={line width=.5pt,dash pattern=on 2pt off 1.5pt}
         }% end of tikzset

%%%%%%%%%%%%%%%%%%%%%%%%%%%%%%%%%%%%%%%%%%%%%%%%%%%%%%%%%%%%%%%%%%%%%%%%%%%%%%

\begin{document}
\centering


\begin{tikzpicture}[>=stealth]

\node [inner sep=0pt] at (1, 0) {\includegraphics[height=7cm]{../AFM-pol-300}};
\node [inner sep=0pt] at (9.5, 0) {\includegraphics[height=7cm]{../FM-pol-300}};

\node [inner sep=0pt] at (5, -11) {\includegraphics[height=15cm]{../figS4}};

\node at (-1.5, 2.5){\Huge \textbf{a}};
\node at (7.0, 2.5){\Huge \textbf{b}};

\node at (-3.25, -3.5){\Huge \textbf{c}};


\end{tikzpicture}


\end{document}
%%%%%%%%%%%%%%%%%%%%%%%%%%%%%%%%%%%%%%%%%%%%%%%%%%%%%%%%%%%%%%%%%%%%%%%%%%%%%%%
}
	\caption{Spin polarization of the extended triangulene obtained with $U=3.0$ eV for fixed \textbf{a} $S_{z}=0$ and \textbf{b} $S_{z}=1$. \textbf{c} Energy difference between the two configurations of \textbf{a,b} for different $U$ values.}
\end{figure*}

\begin{figure*}
	\scalebox{.5}{\documentclass{standalone}

\usepackage{graphicx}
\usepackage{hyperref}
\usepackage{epsfig}
\usepackage{color}
\usepackage{psfrag}
\usepackage{lpic}
\usepackage{tikz}
\usetikzlibrary{decorations.pathmorphing}

\tikzset{
dash/.style={line width=.5pt,dash pattern=on 2pt off 1.5pt}
         }% end of tikzset

%%%%%%%%%%%%%%%%%%%%%%%%%%%%%%%%%%%%%%%%%%%%%%%%%%%%%%%%%%%%%%%%%%%%%%%%%%%%%%

\begin{document}
\centering


\begin{tikzpicture}[>=stealth]

\node [inner sep=0pt] at (0, 0) {\includegraphics[height=18cm]{../figS5}};


\node [inner sep=0pt] at (14, 5.5) {\includegraphics[height=6cm]{../H-passivation/pos-1/pol-300}};
\node [inner sep=0pt] at (22, 5.5) {\includegraphics[height=6cm]{../H-passivation/pos-2/pol-300}};
\node [inner sep=0pt] at (14, -0.5) {\includegraphics[height=6cm]{../H-passivation/pos-3/pol-300}};
\node [inner sep=0pt] at (22, -0.5) {\includegraphics[height=6cm]{../H-passivation/pos-4/pol-300}};
\node [inner sep=0pt] at (18, -6.5) {\includegraphics[height=6cm]{../H-passivation/pos-5/pol-300}};

%%%%%%%%%%%%%%%%%%%%%%%%%%%%%%%%%%%%% Draw symmetry axis %%%%%%%%%%%%%%%%%%%%%%%%%%%

\node at (-11, 7.35){\Huge \textbf{a}};
\node at (11.75, 7.7){\Huge \textbf{b}};
\node at (19.75, 7.7){\Huge \textbf{c}};
\node at (11.75, 1.7){\Huge \textbf{d}};
\node at (19.75, 1.7){\Huge \textbf{e}};
\node at (15.75, -4.3){\Huge \textbf{f}};

\end{tikzpicture}


\end{document}
%%%%%%%%%%%%%%%%%%%%%%%%%%%%%%%%%%%%%%%%%%%%%%%%%%%%%%%%%%%%%%%%%%%%%%%%%%%%%%%
}
	\caption{\textbf{a-e} Spin polarization of the extended molecule with H-passivated C-atom sites in five different positions obtained with $U=3.0$ eV. \textbf{f} Total energy difference between the molecules from \textbf{a-e} figures compared to the molecule from \textbf{a} for different $U$ values.}
\end{figure*}

\begin{figure*}
	\scalebox{.4}{\documentclass{standalone}

\usepackage{graphicx}
\usepackage{hyperref}
\usepackage{epsfig}
\usepackage{color}
\usepackage{psfrag}
\usepackage{lpic}
\usepackage{tikz}
\usetikzlibrary{decorations.pathmorphing}

\tikzset{
dash/.style={line width=.5pt,dash pattern=on 2pt off 1.5pt}
         }% end of tikzset

%%%%%%%%%%%%%%%%%%%%%%%%%%%%%%%%%%%%%%%%%%%%%%%%%%%%%%%%%%%%%%%%%%%%%%%%%%%%%%

\begin{document}
\centering


\begin{tikzpicture}[>=stealth]

\node [inner sep=0pt] at (0, 0) {\includegraphics[height=10cm]{../H-passivation/dimer-pentagon/2H-pos-1-2/pol-U300}};
\node [inner sep=0pt] at (14, 0) {\includegraphics[height=10cm]{../H-passivation/dimer-pentagon/pos-1/pol-U300}};
\node [inner sep=0pt] at (28, 0) {\includegraphics[height=10cm]{../H-passivation/dimer-pentagon/pos-2/pol-U300}};

\node [inner sep=0pt] at (0, -10) {\includegraphics[height=10cm]{../H-passivation/dimer-pentagon/2H-pos-1-2/U300_DOS}};
\node [inner sep=0pt] at (14, -10) {\includegraphics[height=10cm]{../H-passivation/dimer-pentagon/pos-1/U300_DOS}};
\node [inner sep=0pt] at (28, -10) {\includegraphics[height=10cm]{../H-passivation/dimer-pentagon/pos-2/U300_DOS}};


%%%%%%%%%%%%%%%%%%%%%%%%%%%%%%%%%%%%% Draw symmetry axis %%%%%%%%%%%%%%%%%%%%%%%%%%%

\node at (-4.65, 3.35){\Huge \textbf{a}};
\node at (9.25, 3.35){\Huge \textbf{b}};
\node at (23, 3.35){\Huge \textbf{c}};

\node at (-4.65, -5.75){\Huge \textbf{d}};
\node at (9.25, -5.75){\Huge \textbf{e}};
\node at (23, -5.75){\Huge \textbf{f}};


\end{tikzpicture}


\end{document}
%%%%%%%%%%%%%%%%%%%%%%%%%%%%%%%%%%%%%%%%%%%%%%%%%%%%%%%%%%%%%%%%%%%%%%%%%%%%%%%
}
	\caption{\textbf{a,b,c} Spin polarization of the triangulene-dimer molecule H-passiated sites in different positions. \textbf{d,e,f} Density of states of the molecules from the top row obtained at the enery of the HOMO for each case. A smearing parameter $\eta=1$ meV was used to obtain the DOS as a Lorentzian distribution. Used $U=3.0$ eV in all calculations presented in \textbf{a-f}.} 
\end{figure*}

\begin{figure*}
	\scalebox{.6}{\documentclass{standalone}

\usepackage{graphicx}
\usepackage{hyperref}
\usepackage{epsfig}
\usepackage{color}
\usepackage{psfrag}
\usepackage{lpic}
\usepackage{tikz}
\usetikzlibrary{decorations.pathmorphing}

\tikzset{
dash/.style={line width=.5pt,dash pattern=on 2pt off 1.5pt}
         }% end of tikzset

%%%%%%%%%%%%%%%%%%%%%%%%%%%%%%%%%%%%%%%%%%%%%%%%%%%%%%%%%%%%%%%%%%%%%%%%%%%%%%

\begin{document}
\centering


\begin{tikzpicture}[>=stealth]
\node [inner sep=0pt] at (0., 0) {\includegraphics[height=10cm]{../dimer-pentagon/U0_spectrum}};
\node [inner sep=0pt] at (0, -10.0) {\includegraphics[height=10cm]{../dimer-pentagon/U300_spectrum}};

\node at (14.5, 5.7){\Huge $U=0$};

\node [inner sep=0pt] at (11.3, 2) {\includegraphics[height=6cm]{../dimer-pentagon/U0_state35_up}};
\node [inner sep=0pt] at (18.5, 2) {\includegraphics[height=6cm]{../dimer-pentagon/U0_state36_up}};

\node at (14.5, -2.2){\Huge $U=3.0$ eV};

\node [inner sep=0pt] at (11.3, -6) {\includegraphics[height=6cm]{../dimer-pentagon/U300_state36_up}};
\node [inner sep=0pt] at (18.5, -6) {\includegraphics[height=6cm]{../dimer-pentagon/U300_state37_up}};
\node [inner sep=0pt] at (11.3, -12) {\includegraphics[height=6cm]{../dimer-pentagon/U300_state34_dn}};
\node [inner sep=0pt] at (18.5, -12) {\includegraphics[height=6cm]{../dimer-pentagon/U300_state35_dn}};

%%%%%%%%%%%%%%%%%%%%%%%%%%%%%%%%%%%%% Draw symmetry axis %%%%%%%%%%%%%%%%%%%%%%%%%%%

\node at (-6.25, 5.35){\huge \textbf{a}};
\node at (-6.25, -5.35){\huge \textbf{b}};

\node at (8.25, 4.35){\huge \textbf{c}};
\node at (15.45, 4.35){\huge \textbf{d}};

\node at (8.25, -3.75){\huge \textbf{d}};
\node at (15.45, -3.75){\huge \textbf{e}};


\node at (8.25, -9.85){\huge \textbf{g}};
\node at (15.45, -9.85){\huge \textbf{h}};

\end{tikzpicture}


\end{document}
%%%%%%%%%%%%%%%%%%%%%%%%%%%%%%%%%%%%%%%%%%%%%%%%%%%%%%%%%%%%%%%%%%%%%%%%%%%%%%%
}
	\caption{Spatial localization of each state of the trangulene-dimer molecule computed as $\eta_{\alpha\sigma}=\int dr|\psi_{\alpha\sigma}|^{4}$ obtained with \textbf{a} $U=0$ and \textbf{b} $U=3.0$ eV. \textbf{c, d} HOMO and LUMO wavefunction spatial distributions for $U=0$. \textbf{e-h} HOMO and LUMO spatial distributions for \emph{up}- and \emph{down}-electrons obtained with $U=3.0$ eV.} 
\end{figure*}

\begin{figure*}
	\scalebox{.4}{\documentclass{standalone}

\usepackage{graphicx}
\usepackage{hyperref}
\usepackage{epsfig}
\usepackage{color}
\usepackage{psfrag}
\usepackage{lpic}
\usepackage{tikz}
\usetikzlibrary{decorations.pathmorphing}

\tikzset{
dash/.style={line width=.5pt,dash pattern=on 2pt off 1.5pt}
         }% end of tikzset

%%%%%%%%%%%%%%%%%%%%%%%%%%%%%%%%%%%%%%%%%%%%%%%%%%%%%%%%%%%%%%%%%%%%%%%%%%%%%%

\begin{document}
\centering


\begin{tikzpicture}[>=stealth]

\node [inner sep=0pt] at (-2.55, 0) {\includegraphics[height=10cm]{../dimer-pentagon/AFM-pol-300}};
\node [inner sep=0pt] at (8.25, 0) {\includegraphics[height=10cm]{../dimer-pentagon/FM1-pol-300}};
\node [inner sep=0pt] at (19.1, 0) {\includegraphics[height=10cm]{../dimer-pentagon/FM2-pol-300}};

\node [inner sep=0pt] at (7.5, -14) {\includegraphics[height=19cm]{../dimer-pentagon/figS9}};

\node at (-7.3, 3.75){\Huge \textbf{a}};
\node at (3.55, 3.75){\Huge \textbf{b}};
\node at (14.4, 3.75){\Huge \textbf{c}};
\node at (-4.25, -5.5){\Huge \textbf{d}};


\end{tikzpicture}


\end{document}
%%%%%%%%%%%%%%%%%%%%%%%%%%%%%%%%%%%%%%%%%%%%%%%%%%%%%%%%%%%%%%%%%%%%%%%%%%%%%%%
}
	\caption{Spin polarization of the dimer obtained with $U=3.0$ eV for \textbf{a} $S_{z}=0$, \textbf{b} $S_{z}=1$, \textbf{c} $S_{z}=2$. \textbf{c} Energy difference between the magnetic states presented in \textbf{a, b, c} compared to the ground state ($S_{z}=1$) for different $U$ values. } 
\end{figure*}

\end{document}
