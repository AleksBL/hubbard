\documentclass[amsmath,%draft,
amssymb,prb,superscriptaddress]{revtex4}
\usepackage{graphicx}
\usepackage{standalone}
\usepackage{lpic}
\usepackage{xcolor}
\usepackage{tikz}

\newcommand{\Eqref}[1]{Eq.~(\ref{#1})}
\newcommand{\Figref}[1]{Fig.~\ref{#1}}
\newcommand{\Secref}[1]{Sec.~\ref{#1}}


\begin{document}

\title{Phase diagrams of $(n,m,w)$-chGNR}

\author{S. Sanz}
\affiliation{\mbox{Donostia International Physics Center (DIPC) -- UPV/EHU, 20018 San
Sebasti\'an, Spain}}

\author{T. Frederiksen}
\affiliation{\mbox{Donostia International Physics Center (DIPC) -- UPV/EHU, 20018 San
Sebasti\'an, Spain}}
\affiliation{IKERBASQUE, Basque Foundation for Science, E-48011, Bilbao, Spain}

\date{\today}

\begin{abstract}
We study the band gap of ch-GNRs as a function of the chiral parameters $(n,m)$ for $W=6, \ 8, \ 10, \ 12$. We use the 1NN Tight-Binding model and $U=0$ to describe the system. The band gap is multiplied by $(-1)^{Z_{2}}$, where $Z_{2}$ is the topological invariant obtained as $e^{\sum_{n}\gamma_{n}}$ (with $\gamma_{n}$ the Zak phase of each occupied band). Therefore, the ``positive'' values correspond to topologically trivial systems, while the ``negative'' values correspond to topologically non-trivial systems.% Since $m$ is the number of armchair sites, the maximum value it can take depends on the width as $m\leq w/2$ for the unit cells to be coupled.
\end{abstract}


\maketitle

\begin{figure*}
	\scalebox{2.8}{\documentclass{standalone}

\usepackage{graphicx}
\usepackage{hyperref}
\usepackage{epsfig}
\usepackage{color}
\usepackage{psfrag}
\usepackage{lpic}
\usepackage{tikz}
\usetikzlibrary{decorations.pathmorphing}

\tikzset{
dash/.style={line width=.5pt,dash pattern=on 2pt off 1.5pt}
         }% end of tikzset

%%%%%%%%%%%%%%%%%%%%%%%%%%%%%%%%%%%%%%%%%%%%%%%%%%%%%%%%%%%%%%%%%%%%%%%%%%%%%%

\begin{document}
\centering


\begin{tikzpicture}[>=stealth]

\node [inner sep=0pt] at (0, 0) {\includegraphics[height=4cm]{../2-blocks/AFM-pol-300}};
\node [inner sep=0pt] at (5, 0) {\includegraphics[height=4cm]{../2-blocks/FM-pol-300}};
\node [inner sep=0pt] at (2.2, -5.5) {\includegraphics[height=7cm]{../2-blocks/FM-AFM}};

%\node [inner sep=0pt] at (10, 0) {\includegraphics[height=4cm]{../2-blocks-pentagon/AFM-pol-300}};
%\node [inner sep=0pt] at (15, 0) {\includegraphics[height=4cm]{../2-blocks-pentagon/FM-pol-300}};
%\node [inner sep=0pt] at (12.2, -5.5) {\includegraphics[height=7cm]{../2-blocks-pentagon/FM-AFM}};
%


\node at (-2.2, 2){\large \textbf{a}};
\node at (2.7, 2){\large \textbf{b}};
\node at (-2.2, -2.){\Large \textbf{c}};
%
%\node at (-2.75, -4.75){\Large \textbf{d}};
%\node at (5.75, -4.75){\Large \textbf{e}};
%\node at (14.25, -4.75){\Large \textbf{f}};


\end{tikzpicture}


\end{document}
%%%%%%%%%%%%%%%%%%%%%%%%%%%%%%%%%%%%%%%%%%%%%%%%%%%%%%%%%%%%%%%%%%%%%%%%%%%%%%%
}
	\caption{Three repetitions of the unit cell of a (3,1,8)-chGNR.}
\end{figure*}

\begin{figure*}
	\includegraphics[height=5.2cm]{../W6_band_gap_imshow}
	\includegraphics[height=5.2cm]{../W8_band_gap_imshow}
	\includegraphics[height=5.2cm]{../W10_band_gap_imshow}
	\includegraphics[height=5.2cm]{../W12_band_gap_imshow}
	\caption{Band gap of $(n,m,w)$-GNRs as a function of the chiral parameters $(n,m)$ for $W=6, \ 8, \ 10, \ 12$ in logarithmic scale. The value of the band gap is in each case multiplied by $(-1)^{Z_{2}}$, with $Z_{2}$ the topological invariant. Therefore, the ``positive'' (red) values correspond to topologically trivial systems, while the ``negative'' (blue) values correspond to topologically non-trivial systems.}
\end{figure*}

\end{document}