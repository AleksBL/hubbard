\documentclass[amsmath,%draft,
amssymb,prb,superscriptaddress]{revtex4}
\usepackage{graphicx}
\usepackage{standalone}
\usepackage{lpic}
\usepackage{xcolor}
\usepackage{tikz}

\newcommand{\Eqref}[1]{Eq.~(\ref{#1})}
\newcommand{\Figref}[1]{Fig.~\ref{#1}}
\newcommand{\Secref}[1]{Sec.~\ref{#1}}

\newcommand{\eg}[1]{\textit{e.g.}}

\begin{document}

\title{Bulk-boundary correspondance of $(n,m,w)$-chGNR}

\author{S. Sanz}
\affiliation{\mbox{Donostia International Physics Center (DIPC) -- UPV/EHU, 20018 San
Sebasti\'an, Spain}}

\author{T. Frederiksen}
\affiliation{\mbox{Donostia International Physics Center (DIPC) -- UPV/EHU, 20018 San
Sebasti\'an, Spain}}
\affiliation{IKERBASQUE, Basque Foundation for Science, E-48011, Bilbao, Spain}

\date{\today}

\begin{abstract}
We study the band gaps of ch-GNRs as a function of the chiral parameters $(n,m,w)$. The band gap is multiplied by $(-1)^{Z_{2}}$, where $Z_{2}$ is the topological invariant obtained as $e^{\sum_{n}\gamma_{n}}$ (with $\gamma_{n}$ the Zak phase of each occupied band). We compare the calculations obtained with 1NN and 3NN Tight-Binding models. %Therefore, the ``positive'' values correspond to topologically trivial systems, while the ``negative'' values correspond to topologically non-trivial systems.
We then focus on the experimental cases $n=3, m=1, 2$ and $w=4,6,8$. We also show the behavior of the HOMO and LUMO wavefunctions along the lower edge of a \emph{finite} ribbon of 15 p.u. and the dependency of the HOMO-LUMO gap as a function of the (finite) chGNR length. We also show the bulk density of states (infinite ribbon) and the surface density of states (semi-infinite ribbon). All calculations here were performed with $U=0$. We conclude by saying that since we find small differences between the phase diagrams of \Figref{fig:phase-diagrams} obtained with 1NN and 3NN models, there are some cases where the topoology is not very robust, and hence they are not the best examples to search for topological properties since their qualitative electronic behavior is sensitive to small details. This argument is supported by \Figref{fig:gap-fit-length}, where it can be seen that the size effect is crucial to find the ingap states for the systems that have a very small bandgap (\eg \ \ $(3,1,8)$-chGNR), while the cases that present a large bandgap seem to have more robust topological properties (\eg \ \ $(3,2,8)$-chGNR).
\end{abstract}


\maketitle

%\textbf{Inversion symmetry}:\\
%
%
%The presence of inversion symmetry provides another way to find the topological invariant, which can be determined from the knowledge pf the parity of the occupied band eigenstates at the time-reversal invariant momenta ($\Gamma$ and $X$). This statement has been proven for 2D and 3D topological insulators in ref. \cite{Fu2007}. Summarizing, in \emph{topological} 1D-insulators there will be an odd number of bands that invert their parity with respect to these two $k$-points, while in the trivial insulators there will be an \emph{even} number of bands that will invert their symmetry. 

\begin{figure*}
	\scalebox{1.8}{\documentclass{standalone}

\usepackage{graphicx}
\usepackage{hyperref}
\usepackage{epsfig}
\usepackage{color}
\usepackage{psfrag}
\usepackage{lpic}
\usepackage{tikz}
\usetikzlibrary{decorations.pathmorphing}

\tikzset{
dash/.style={line width=.5pt,dash pattern=on 2pt off 1.5pt}
         }% end of tikzset

%%%%%%%%%%%%%%%%%%%%%%%%%%%%%%%%%%%%%%%%%%%%%%%%%%%%%%%%%%%%%%%%%%%%%%%%%%%%%%

\begin{document}
\centering


\begin{tikzpicture}[>=stealth]

\node [inner sep=0pt] at (0, 0) {\includegraphics[height=4cm]{../2-blocks/AFM-pol-300}};
\node [inner sep=0pt] at (5, 0) {\includegraphics[height=4cm]{../2-blocks/FM-pol-300}};
\node [inner sep=0pt] at (2.2, -5.5) {\includegraphics[height=7cm]{../2-blocks/FM-AFM}};

%\node [inner sep=0pt] at (10, 0) {\includegraphics[height=4cm]{../2-blocks-pentagon/AFM-pol-300}};
%\node [inner sep=0pt] at (15, 0) {\includegraphics[height=4cm]{../2-blocks-pentagon/FM-pol-300}};
%\node [inner sep=0pt] at (12.2, -5.5) {\includegraphics[height=7cm]{../2-blocks-pentagon/FM-AFM}};
%


\node at (-2.2, 2){\large \textbf{a}};
\node at (2.7, 2){\large \textbf{b}};
\node at (-2.2, -2.){\Large \textbf{c}};
%
%\node at (-2.75, -4.75){\Large \textbf{d}};
%\node at (5.75, -4.75){\Large \textbf{e}};
%\node at (14.25, -4.75){\Large \textbf{f}};


\end{tikzpicture}


\end{document}
%%%%%%%%%%%%%%%%%%%%%%%%%%%%%%%%%%%%%%%%%%%%%%%%%%%%%%%%%%%%%%%%%%%%%%%%%%%%%%%
}
	\caption{Example to define the chiral parameters $(n,m,w)$: three repetitions of the unit cell of a (3,1,8)-chGNR.}
\end{figure*}

\begin{figure*}
	\scalebox{1.8}{\documentclass{standalone}

\usepackage{graphicx}
\usepackage{hyperref}
\usepackage{epsfig}
\usepackage{color}
\usepackage{psfrag}
\usepackage{lpic}
\usepackage{tikz}
\usetikzlibrary{decorations.pathmorphing}

\tikzset{
dash/.style={line width=.5pt,dash pattern=on 2pt off 1.5pt}
         }% end of tikzset

%%%%%%%%%%%%%%%%%%%%%%%%%%%%%%%%%%%%%%%%%%%%%%%%%%%%%%%%%%%%%%%%%%%%%%%%%%%%%%

\begin{document}
\centering


\begin{tikzpicture}[>=stealth]


\node [inner sep=0pt] at (0, 0) {\includegraphics[height=7cm]{../2-blocks/U0_spectrum}};
\node [inner sep=0pt] at (0, -7) {\includegraphics[height=7cm]{../2-blocks/U300_spectrum}};

\node [inner sep=0pt] at (7, 1) {\includegraphics[height=3cm]{../2-blocks/U0_state43_up}};
\node [inner sep=0pt] at (12, 1) {\includegraphics[height=3cm]{../2-blocks/U0_state44_up}};

\node [inner sep=0pt] at (7, -4) {\includegraphics[height=3cm]{../2-blocks/U300_state43_up}};
\node [inner sep=0pt] at (12, -4) {\includegraphics[height=3cm]{../2-blocks/U300_state44_up}};

\node [inner sep=0pt] at (7, -8) {\includegraphics[height=3cm]{../2-blocks/U300_state43_dn}};
\node [inner sep=0pt] at (12, -8) {\includegraphics[height=3cm]{../2-blocks/U300_state44_dn}};


\node at (9.2, 3.5){\Large $U=0$ eV};
\node at (9.2, -1.5){\Large $U=3.0$ eV};

\node at (-4, 3.35){\large \textbf{a}};
\node at (-4, -3.75){\large \textbf{b}};

\node at (5.25, 2.75){\large \textbf{c}};
\node at (10.2, 2.75){\large \textbf{d}};

\node at (5.25, -2.2){\large \textbf{d}};
\node at (10.2, -2.2){\large \textbf{e}};


\node at (5.25, -6.2){\large \textbf{g}};
\node at (10.2, -6.2){\large \textbf{h}};

%
%\node at (-2.75, -4.75){\Large \textbf{d}};
%\node at (5.75, -4.75){\Large \textbf{e}};
%\node at (14.25, -4.75){\Large \textbf{f}};


\end{tikzpicture}


\end{document}
%%%%%%%%%%%%%%%%%%%%%%%%%%%%%%%%%%%%%%%%%%%%%%%%%%%%%%%%%%%%%%%%%%%%%%%%%%%%%%%
}
	\caption{Band gap of $(n,m,w)$-GNRs as a function of the chiral parameters $(n,m,w)$ in linear scale obtained with 1NN (top row) and 3NN (bottom row). The value of the band gap is in each case multiplied by $(-1)^{Z_{2}}$, with $Z_{2}$ the topological invariant. Therefore, the ``positive'' (blue) values correspond to topologically trivial systems, while the ``negative'' (red) values correspond to topologically non-trivial systems.}
	\label{fig:phase-diagrams}
\end{figure*}

\begin{figure*}
	\scalebox{1.8}{\documentclass{standalone}

\usepackage{graphicx}
\usepackage{hyperref}
\usepackage{epsfig}
\usepackage{color}
\usepackage{psfrag}
\usepackage{lpic}
\usepackage{tikz}
\usetikzlibrary{decorations.pathmorphing}

\tikzset{
dash/.style={line width=.5pt,dash pattern=on 2pt off 1.5pt}
         }% end of tikzset

%%%%%%%%%%%%%%%%%%%%%%%%%%%%%%%%%%%%%%%%%%%%%%%%%%%%%%%%%%%%%%%%%%%%%%%%%%%%%%

\begin{document}
\centering


\begin{tikzpicture}[>=stealth]

\node [inner sep=0pt] at (-0.5, 0) {\includegraphics[height=4cm]{../3-blocks/AFM-pol-300}};
\node [inner sep=0pt] at (5.5, 0) {\includegraphics[height=4cm]{../3-blocks/FM-pol-300}};
\node [inner sep=0pt] at (2.2, -5.5) {\includegraphics[height=7cm]{../3-blocks/FM-AFM}};

%\node [inner sep=0pt] at (10, 0) {\includegraphics[height=4cm]{../3-blocks-pentagon/AFM-pol-300}};
%\node [inner sep=0pt] at (15, 0) {\includegraphics[height=4cm]{../3-blocks-pentagon/FM-pol-300}};
%\node [inner sep=0pt] at (12.2, -5.5) {\includegraphics[height=7cm]{../3-blocks-pentagon/FM-AFM}};
%
%

\node at (-2.6, 2){\large \textbf{a}};
\node at (3.2, 2){\large \textbf{b}};
\node at (-2.2, -2.){\Large \textbf{c}};

%\node at (-2.75, -4.75){\Large \textbf{d}};
%\node at (5.75, -4.75){\Large \textbf{e}};
%\node at (14.25, -4.75){\Large \textbf{f}};
%

\end{tikzpicture}


\end{document}
%%%%%%%%%%%%%%%%%%%%%%%%%%%%%%%%%%%%%%%%%%%%%%%%%%%%%%%%%%%%%%%%%%%%%%%%%%%%%%%
}
	\caption{Bands structure of $(3,1,w)$-chGNRs for $w=4,6,8$ (black, blue, red lines respectively), obtained with 1NN model (left) and 3NN model (right).}
\end{figure*}

\begin{figure*}
	%\includegraphics[height=3.5cm]{../../ssh/SSH-trivial_bands}
	%\includegraphics[height=3.5cm]{../../ssh/SSH-topological_bands}\\
	\includegraphics[height=6cm]{../3-1-4/bands_1NN}
	\includegraphics[height=6cm]{../3-1-6/bands_1NN}
	\includegraphics[height=6cm]{../3-1-8/bands_1NN}\\
	\includegraphics[height=6cm]{../3-2-4/bands_1NN}
	\includegraphics[height=6cm]{../3-2-6/bands_1NN}
	\includegraphics[height=6cm]{../3-2-8/bands_1NN}
	\caption{Band structure of the (n,m,w)-chGNRs obtained with 1NN model. The topological invariant obtained from the Zak phase calculation is annotated in a black rectangle. }
\end{figure*}

\begin{figure*}
	%\includegraphics[height=4cm]{../../ssh/SSH-trivial_edge_wf}
	%\includegraphics[height=4cm]{../../ssh/SSH-topological_edge_wf}\\
	\includegraphics[height=4cm]{../3-1-4/1NN_edge_wf}
	\includegraphics[height=4cm]{../3-1-6/1NN_edge_wf}
	\includegraphics[height=4cm]{../3-1-8/1NN_edge_wf}
	\includegraphics[height=4cm]{../3-2-4/1NN_edge_wf}
	\includegraphics[height=4cm]{../3-2-6/1NN_edge_wf}
	\includegraphics[height=4cm]{../3-2-8/1NN_edge_wf}
	\caption{Behavior of the HOMO and LUMO wavefunctions at the lower edge of the \emph{finite} ribbons of 15 precursor units obtianed with the 1NN model.}
\end{figure*}

\begin{figure*}
	%\includegraphics[height=4cm]{../../ssh/SSH-trivial_gap_fit}
	%\includegraphics[height=4cm]{../../ssh/SSH-topological_gap_fit}
	\includegraphics[height=4cm]{../3-1-4/1NN_gap_fit}
	\includegraphics[height=4cm]{../3-1-6/1NN_gap_fit}
	\includegraphics[height=4cm]{../3-1-8/1NN_gap_fit}
	\includegraphics[height=4cm]{../3-2-4/1NN_gap_fit}
	\includegraphics[height=4cm]{../3-2-6/1NN_gap_fit}
	\includegraphics[height=4cm]{../3-2-8/1NN_gap_fit}	
	\caption{HOMO-LUMO gap as a function of the ribbon length fitted with two possible functions $f_{1}(L) \propto L^{-b}$ and $f_{2}(L) \propto e^{-\alpha L}$ obtained with the 1NN model.}	
	\label{fig:gap-fit-length}		
\end{figure*}

\begin{figure*}
	%\includegraphics[height=4cm]{../../ssh/SSH-trivial_DOS}
	%\includegraphics[height=4cm]{../../ssh/SSH-topological_DOS}\\
	\includegraphics[height=4cm]{../3-1-4/1NN_DOS_U0}
	\includegraphics[height=4cm]{../3-1-6/1NN_DOS_U0}
	\includegraphics[height=4cm]{../3-1-8/1NN_DOS_U0}
	\includegraphics[height=4cm]{../3-2-4/1NN_DOS_U0}
	\includegraphics[height=4cm]{../3-2-6/1NN_DOS_U0}
	\includegraphics[height=4cm]{../3-2-8/1NN_DOS_U0}
	\caption{Bulk and surface Density of states of the $(n,m,w)$-chGNRs. The DOS were obtained as $DOS_{\nu}=-\frac{1}{\pi}\mathrm{Im}\left[\mathrm{Tr}\left[ g_{\nu} \right]\right]$, where $g$ is the Green's function and $\nu$ denotes the surface or bulk.}		
\end{figure*}

%\bibliography{/media/sofia/usb/literature/Literature} 

\end{document}